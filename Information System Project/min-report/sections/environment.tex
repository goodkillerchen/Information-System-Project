The detailed information of programming are listed in Table.\ref{tab:environment}.
\begin{table}[!hpbt]
	\centering
	\begin{tabular}{|l|l|}
	\hline
	Attribute & Content  \\ \hline
	Operation System & \textbf{Windows SDK edition: 10.0}  \\ \hline
	Integrated Development Environment  &  \textbf{Visual Studio 2019(v142)}    \\ \hline
	Solution Settings &  \textbf{Release in x86 Platform}\\\hline
	Optimization  &  \textbf{O2 optimize} \\ \hline
	Programming Language &  \textbf{C\#} \\ \hline
	Framework &  \textbf{. Net framework 4.7.2}\\ \hline
	Source Code hosting Platform &  \textbf{Github} \\ \hline
	\end{tabular}
	\caption{Settings and Attribute}
	\label{tab:environment}
\end{table}

In order to make the operation more smooth, 
all the program environment and settings are completed and included in the file package. 
Just required to follow the steps below to run the program.
\begin{enumerate}
\item Open the package and find the \emph{Information System Project.exe} file. Double-click the file to enter the program interface as Figure.\ref{fig:init} exactly.
\begin{figure}[!htbp]
	\centering
	\includegraphics[width=0.75\textwidth]{images/initial.png}
	\caption{Initial GUI}
	\label{fig:init}
\end{figure}

In order to record the relevant output data of the program and facilitate display and modification later. 
It is required to create a \emph{.mdb} file to store it, which is called \emph{DataBase.mdb} in project package for example.

\item Choose the data of each parameter and input on the program surface as Figure.\ref{fig:st1}.
\begin{figure}[!htbp]
	\centering
	\includegraphics[width=0.75\textwidth]{images/step1.png}
	\caption{Step1}
	\label{fig:st1}
\end{figure}

\item Choose the DB file to store and operate the data, click the button \textbf{File} and choose the \emph{.mdb} as Figure.\ref{fig:st2}
In the previous step and \textbf{Confirm} if all get right.(\textbf{Clear} is a function that clear all the data you have input, 
including the parameter followed \textup{Figure.\ref{fig:st3}})
\begin{figure}[!htbp]
	\centering
	\includegraphics[width=0.75\textwidth]{images/step2.png}
	\caption{Step2}
	\label{fig:st2}
\end{figure}
\begin{figure}[!htbp]
	\centering
	\includegraphics[width=0.75\textwidth]{images/step3.png}
	\caption{Step3}
	\label{fig:st3}
\end{figure}

\item Push the \textbf{RUN} button and the \textbf{N} number and final answer of your input will be shown 
on the surface window as Figure.\ref{fig:st4}, you can check the answer after that.
\begin{figure}[!htbp]
	\centering
	\includegraphics[width=0.75\textwidth]{images/step4.png}
	\caption{Step4}
	\label{fig:st4}
\end{figure}

\item After confirming the data is correct, use the button \textbf{Insert to database}
(Figure.\ref{fig:st5}) to download the data on the DB file(\emph{\*.mdb}), and the button \textbf{Open the file}(Figure.\ref{fig:st6}) can open it to display the data you have calculated. 
It is also easy for you to delete or use any other operation on the data though your DB file.
\begin{figure}[!htbp]
	\centering
	\includegraphics[width=0.75\textwidth]{images/step5.png}
	\caption{Step5}
	\label{fig:st5}
\end{figure}
\begin{figure}[!htbp]
	\centering
	\includegraphics[width=0.75\textwidth]{images/step6.png}
	\caption{Step6}
	\label{fig:st6}
\end{figure}

\item After adding the data into your DB file, you can get the record of the message 
including order , calculated result and data information, which you can see the tips of first row.
Moreover, you can choose the data which you plan to delete by clicking the right button of mouse, 
then you can see the menu item \textbf{Delete} (Figure.\ref{fig:st7}). 
\begin{figure}[!htbp]
	\centering
	\includegraphics[width=0.75\textwidth]{images/step7.png}
	\caption{Step7}
	\label{fig:st7}
\end{figure}

\item Besides, you can clear all the data from your DB file by using button \textbf{Reset} (Figure.\ref{fig:st8}). 
\begin{figure}[!htbp]
	\centering
	\includegraphics[width=0.75\textwidth]{images/step8.png}
	\caption{Step8}
	\label{fig:st8}
\end{figure}
\end{enumerate}