
We can divide the problem into two parts, $j=s$ and $j \neq s$.
\subsection{$j=s$}
    
    \subsubsection{Algorithm to Find Subsets}
    Now, we have a set whose number of the element is $n$. Then we want to find out all the subsets whose 
    number of the element is $k$.

    \textbf{Algorithm: }
        \begin{itemize}
            \item First, we put the origin set to a container, and then we label every element to one(illustrate the picture below). 
            We assume that the origin set is $S$, $S=\{1,2,3,4,5\}$ in Table.\ref{tab:S}.
            \begin{table}[!hpbt]
                \centering
                \begin{tabular}{lllll}
                \hline
                \multicolumn{1}{|l|}{5} & \multicolumn{1}{l|}{4} & \multicolumn{1}{l|}{3} & \multicolumn{1}{l|}{2} & \multicolumn{1}{l|}{1} \\ \hline
                1                       & 1                      & 1                      & 1                      & 1                     
                \end{tabular}
                \caption{}
                \label{tab:S}
            \end{table}
            Then, the subset which has the same element with the original set's is labeled the element to 1, otherwise labeling it to 0. For 
            example, we suppose that one the subset is $S_1$,$S_1=\{1,2,4\}$. We can represent it as Table.\ref{tab:S1}.
            \begin{table}[!hpbt]
                \centering
                \begin{tabular}{lllll}
                \hline
                \multicolumn{1}{|l|}{5} & \multicolumn{1}{l|}{4} & \multicolumn{1}{l|}{3} & \multicolumn{1}{l|}{2} & \multicolumn{1}{l|}{1} \\ \hline
                0                       & 1                      & 0                      & 1                      & 1                     
                \end{tabular}
                \caption{}
                \label{tab:S1}
            \end{table}
            Now we can change the number below the array to a binary number, which means that each subset can be represented by a unique number from 
            0(empty set) to $2^n-1$(original set). Just like the example above set $S$ can be represented by $11111_2 = 31_{10}$ and $S_1$ can be expressed 
            as $01011_2 = 11_{10}$
            \item Subsequently, we know how to find subsets of the original set, but I want to know how to find the subset with the specific number of elements. 
            Therefore, we only need to know the subset whose binary number representation contains $k$ 1s. As the example in Table.\ref{tab:S1}, $S_1=\{1,2,4\}$:
            % \begin{table}[!hpbt]
            %     \centering
            %     \begin{tabular}{lllll}
            %     \hline
            %     \multicolumn{1}{|l|}{5} & \multicolumn{1}{l|}{4} & \multicolumn{1}{l|}{3} & \multicolumn{1}{l|}{2} & \multicolumn{1}{l|}{1} \\ \hline
            %     0                       & 1                      & 0                      & 1                      & 1                     
            %     \end{tabular}
            %     \caption{Factors and Credit}
            %     \label{tab:}
            % \end{table}
            So, the $S_1$ contains three elements, because it has three \emph{1}s.

            In this way, we can easily find out the subset whose number of elements is $k$ from $0$ to $2^n-1$, the code block \hyperref[code:fS1]{findSubsetOfk} illustrates the situation.
            \lstset{xleftmargin=0.5em,xrightmargin=0.5em}
            \label{code:fS1}
            \lstinputlisting[language=c++]{code/findSubset1.cpp}

            However, we can easily find that the binary number representation of the subset whose number of elements is $k$ is 
            no less than $2^k-1$. Therefore, we the code above, we can have an optimization on the $i$. The optimized code \hyperref[code:fS2]{findSubsetOfkOptim} is 

            \lstset{xleftmargin=0.5em,xrightmargin=0.5em}
            \label{code:fS2}
            \lstinputlisting[language=c++]{code/findSubset2.cpp}
            
        \end{itemize}
        \begin{itemize}
            \item Currently, we can use the same way what we say above to find out the subset of the set whose number of element is 
            $k$ and its number of elements is $s$.
        \end{itemize}
    \subsubsection{Calculate the Combination Number}
    If we calculate the combination number directly, it is likely to out of bounds of int. So we can use \hyperref[fol:C]{combination formula}:
    \label{fol:C}$$C_n^m=C_{n-1}^{m-1}+C_{n-1}^m$$ 
    to calculate the combination number. And the specific implementation code can be seen in \hyperref[code:calComb]{calculateCombination}.
   \label{code:calComb}
   \lstinputlisting[language=c++]{code/calculateCombination.cpp}
    
    \subsubsection{Greedy Algorithm to Calculate the Set Coveraged}
    We denote that the input is a set $\mathcal{U}$ of n elements, and a collection $S=\{S_1, S_2, . . . , S_m\}$ of $m$ subsets of $\mathcal{U}$ such that
    $\cup_{i} S_i=\mathcal{U}$. Our goal is to take as few subsets as possible from $S$ such that their union covers $\mathcal{U}$.
    We can solve this problem easily by greedy algorithm. The algorithm is below in Table.\ref{tab:Greedy Cover}:
    \begin{table}[!hpbt]
        \centering
        \begin{tabular}{|l|}
        \hline
        Greedy Cover($S$,$\mathcal{U}$)                                       \\
        1. repeat                                                             \\
        2. pick the set that covers the maximum number of uncover element \\
        3. mark elements in the chosen set as covered                     \\
        4. remove the set from $S$ to the result set                      \\
        5. done                                                               \\ \hline
        \end{tabular}
        \caption{Greedy Cover}
        \label{tab:Greedy Cover}
    \end{table}

Based on the three lemmas above, we can easily transform the problem to that the set $\mathcal{U}=\{1,2,\cdots, C^j_n\}$, which means that we map each different subset 
whose the number of the elements is j to a unique code from 1 to $ C^j_n$. Each subset of $S$, represents the each $k$ set's subsets whose number of elements is j. Ultimately, 
we can solve the problem easily.
\subsection{$j \neq s$}
    The way to solve the problem is just like the way we mentioned above. However, after finishing finding the subset of the $k$ set whose element number is $s$, we should know how many sets 
    whose the number of elements is $j$ include it. Therefore, we use \textbf{DFS(depth first search)} to find out them.
    Assuming that $n=5, s=3, j=4$, and the subset whose number of elements is equal to 3 is labeled as $01011_2$. Therefore, we can expand it as below in Table.\ref{tab:j neq s}.
    \begin{table}[!hpbt]
        \centering
        \begin{tabular}{lllll}
        \hline
        \multicolumn{1}{|l|}{5} & \multicolumn{1}{l|}{4} & \multicolumn{1}{l|}{3} & \multicolumn{1}{l|}{2} & \multicolumn{1}{l|}{1} \\ \hline
        0                       & 1                      & 0                      & 1                      & 1                      \\
        0                       & 1                      & 1                      & 1                      & 1                      \\
        1                       & 1                      & 0                      & 1                      & 1                     
        \end{tabular}
        \caption{}
        \label{tab:j neq s}
    \end{table}
    
    Then, we should mark the last two rows of the set above in the $\mathcal{U}$ as covered.